\documentclass{article}

% Language setting
% Replace `english' with e.g. `spanish' to change the document language
\usepackage[english,russian]{babel}
\usepackage{amsmath}

%графика
\usepackage{wrapfig}
\usepackage{graphicx}
\usepackage{pgfplots}
\usepackage{tikz}


\usepackage{tcolorbox}

% Set page size and margins
% Replace `letterpaper' with `a4paper' for UK/EU standard size
\usepackage[letterpaper,top=2cm,bottom=2cm,left=3cm,right=3cm,marginparwidth=1.75cm]{geometry}

% Useful packages
\usepackage{amsmath}
\usepackage{amssymb}
\usepackage{graphicx}
\usepackage{fixltx2e}
\usepackage[colorlinks=true, allcolors=blue]{hyperref}

\usepackage{geometry}
\geometry{left=25mm,right=25mm,
 top=25mm,bottom=25mm}
\title{Data Science.\\
Лекции. Неделя 3 - 4. \\
Волатильность Доходности и Корреляция.}
\author{Айвазова Кристина}

% Колонтитулы
\usepackage{fancyhdr}
\pagestyle{fancy}
\renewcommand{\headrulewidth}{0.1mm}  
\renewcommand{\footrulewidth}{0.1mm}
\lfoot{}
\rfoot{\thepage}
\cfoot{}
\rhead{CMF-2022}
\chead{}

\begin{document}
\maketitle

% Оглавление
\setcounter{tocdepth}{1} % {2} - в оглавлении участвуют chapter, section и subsection. {1} - только chapter и section
\renewcommand\contentsname{Contents}
\tableofcontents
\newpage

% \section{Dictionary, Definitions, Abbreviations}

% \subsection{Dictionary}
% \begin{itemize}
%     \item IR - Interest rate - процентная ставка.
%     \item Compounding - платежи (idk)
% \end{itemize}

% \subsection{Definitions and Abbreviations}
% \begin{itemize}
%     \item SAR - Stated annual rate.
%     \item EAR - Effective annual rate.
%     \item FoC - Frequency of Compounding
%     \item PMT - Payment
%     \item r - Interest rate (at the moment). 
% \end{itemize}

\renewcommand{\labelitemi}{\tiny$\bullet$}
\renewcommand{\figurename}{Fig.}


\section{Измерение доходности}
\begin{itemize}
  \item Простая доходность активов: \(R_t = \dfrac{(P_t + P_0)}{P_0}\)
  \item Для нескольких периодов: \((1+R_t) = \prod_{t=1}^T(1+R_t)\)
  \item Составная доходность (логарифмическая доходность): \(r_t = \log (P_t) + \log (P_0)\)
  \item Обратите внимание на то, что : \(1+R_t = e^{r_t}\)
  \item Логарифмическая доходность менее интуитивная (например, 100\% убытка соответствует тому же при простой доходности, но примерно 63\% при логарифмической)
\end{itemize}

\section{Волатильность и риск}
\begin{itemize}
  \item Волатильность финансового актива обычно измеряется стандартным отклонением его доходности
  \item Предположим, что \(r_t = \mu + \sigma*\varepsilon_t \) с  имеет нулевое среднее значение и дисперсию 1, часто обозначается \N(0,1)
  \item Поскольку логарифмическая доходность является аддитивной, для n периодов \(r_nt = n\mu + \sigma\sum_{k=1}^{n}\varepsilon_kt\);предполагая, что среднее значение независимости и шкала дисперсии линейны
  \item Например,если среднее значение за день равно \(\mu\), а дисперсия \(\sigma^2\), то среднее значение за неделю равно \(5\mu\) и стандартное отклонение \(\sqrt{5}\sigma\).
\end{itemize}

\section{Подразумеваемая волатильность}
\begin{itemize}
  \item Подразумеваемая волатильность - это альтернативный показатель, который рассчитывается с использованием цен опционов.
  \item Модель Блэка-Шоулза-Мертона связывает цену опциона "колл" с волатильностью как \(C_t = f(r_t, T, P_t, K, \sigma^2)\), где \(r_t\) - безрисковая процентная ставка, $T$ - срок погашения опциона (в годах), $K$ - цена исполнения.
  \item В модели наблюдаемы все значения, кроме волатильности; решение этого уравнения относительно \(\sigma^2\) дает нам подразумеваемое значение волатильности.
  \item Полученная волатильность рассчитана на один год и считается постоянной до истечения срока действия.
  \item VIX - это еще один показатель подразумеваемой волатильности.
  \item Он сочетает в себе опционы для SP500 с различными ценами исполнения на 30 дней
  \item Наиболее важным ограничением VIX является то, что он может быть рассчитан только для активов с крупными ликвидными рынками деривативов.
\item VIX - это прогнозный показатель волатильности по сравнению с оценками волатильности в обратном направлении, полученными с использованием (исторической) доходности активов
\end{itemize}

\section{Распределение доходности}
\begin{itemize}
\itemСтатистика теста Жарке-Бера используется для формальной проверки того, совместимы ли асимметрия выборки и эксцесс с предположением о нормальном распределении результатов.
\item Общая черта многих рядов финансовой доходности - доходности, рассчитанные за более длительные периоды, по-видимому, лучше аппроксимируются нормальным распределением
\item Хвосты распределения определяются как поведение $P(X>x)$ и $P(X<-x)$
для больших $x$
\item Хвост степенного закона - $P(X > x) = ax-b$, $a>0$, $b>0$
\item Тяжелые хвосты (например, хвост степенного закона) приводят к тому, что большие потери появляются чаще, чем при нормальном распределении
\end{itemize}

\section{Корреляция и зависимость}
\begin{itemize}
\item Ковариация двух случайных величин определяется как \(Cov(X,Y) = EXY - EXEY\).
\item Корреляция определяется как \(Corr (X, Y) = \frac{Cov(X,Y)}{\sqrt{Var(X)}\sqrt{Var(Y}}\).
\item Две переменные определяются как статистически независимые, если знание об одной из них не влияет на распределение вероятности для
другой \(f(Y|X = x) = f (Y)\), где $f$ обозначает плотность.
\item Если две случайные величины независимы, их корреляция равна нулю, однако обратное неверно (например, нулевая корреляция не всегда означает независимость).
\item Обратите внимание, что корреляция описывает только линейную зависимость.
\end{itemize}

\section{Коэффициент корреляции Спирмена}
\begin{itemize}
\item Определяется как коэффициент корреляции Пирсона между ранжированными переменными.
\item Ранг - это индекс данных в упорядоченном наборе.
\item Ранговая корреляция Спирмена не зависит от распределения данных и является надежной в отношении выбросов.
\item Если все ранги являются различными целыми числами, то \(r_s = 1 - \frac{6\prod_{i=1}^n{d_i^2}}{n(n^2-1)}\), где \(d_i = rg(X_i) - rg(Y_i)\).
\item Для линейной зависимости Корреляция Спирмена приблизительно равна корреляции Пирсона.
\item Корреляция Спирмена инвариантна относительно нелинейных монотонных преобразований.
\end{itemize}

\section{Ранговая корреляция Кендалла}
\begin{itemize}
\item $\tau$ Кендалла - это показатель ранговой корреляции: сходство порядка данных при ранжировании по каждой из величин.
\item Предположим, что наблюдения $X$ и $Y$ уникальны. Пары наблюдений $(X_i, X_j)$ и $(Y_i, Y_j)$ согласуются, если $X_i > X_j$, $Y_i > Y_j$ или $X_i < X_j$, $Y_i < Y_j$. 
\item Определите \(\tau = \frac{n_c - n_d}{\frac{n(n-1)}{2}}\), где $n_c$ - количество согласующихся пар, $n_d$ - количество несогласованных пар.
\item Альтернативно \(\tau = \frac{1}{n(n-1)}\sum_{i=j}sgn(X_i - X_j)sgn(Y_i - Y_j)\).
\item Монотонные возрастающие преобразования также не влияют на
\(\tau\) Кендалла.
\end{itemize}
\end{document}
