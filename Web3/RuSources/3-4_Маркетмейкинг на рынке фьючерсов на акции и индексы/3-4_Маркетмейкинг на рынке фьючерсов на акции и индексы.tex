\documentclass{article}

\usepackage[english,russian]{babel}
\usepackage{amsmath}

%графика
\usepackage{wrapfig}
\usepackage{graphicx}
\usepackage{pgfplots}
\usepackage{tikz}


\usepackage{tcolorbox}

% Set page size and margins
% Replace `letterpaper' with `a4paper' for UK/EU standard size
\usepackage[letterpaper,top=2cm,bottom=2cm,left=3cm,right=3cm,marginparwidth=1.75cm]{geometry}

% Useful packages
\usepackage{amsmath}
\usepackage{amssymb}
\usepackage{graphicx}
\usepackage{fixltx2e}
\usepackage[colorlinks=true, allcolors=blue]{hyperref}

\usepackage{geometry}
\geometry{left=25mm,right=25mm,
 top=25mm,bottom=25mm}

\title{Web3.\\
Лекция. Неделя 3-4. \\
Маркетмейкинг на рынке фьючерсов на акции и индексы.}
\author{Смирнов Максим}

% Колонтитулы
\usepackage{fancyhdr}
\pagestyle{fancy}
\renewcommand{\headrulewidth}{0.1mm}  
\renewcommand{\footrulewidth}{0.1mm}
\lfoot{}
\rfoot{\thepage}
\cfoot{}
\rhead{CMF-2022}
\chead{}

\begin{document}
\maketitle

% Оглавление
\setcounter{tocdepth}{1} % {2} - в оглавлении участвуют chapter, section и subsection. {1} - только chapter и section
\renewcommand\contentsname{Contents}
\tableofcontents
\newpage

% \section{Dictionary, Definitions, Abbreviations}

% \subsection{Dictionary}
% \begin{itemize}
%     \item IR - Interest rate - процентная ставка.
%     \item Compounding - платежи (idk)
% \end{itemize}

% \subsection{Definitions and Abbreviations}
% \begin{itemize}
%     \item SAR - Stated annual rate.
%     \item EAR - Effective annual rate.
%     \item FoC - Frequency of Compounding
%     \item PMT - Payment
%     \item r - Interest rate (at the moment). 
% \end{itemize}

\renewcommand{\labelitemi}{\tiny$\bullet$}
\renewcommand{\figurename}{Fig.}

\newpage


\section{Введение в маркетмейкинг: на чем зарабатывают обменные пункты валют}
\begin{itemize}
\item \emph{На чем зарабатывают обменные пункты валют?} На спреде (\textbf{Spread}) при взаимодействии продавца и покупателя валюты. Принцип: покупаешь дешево – продаешь дорого. Поскольку количество таких сделок может быть довольно большим, а выставляемый спред достигать 0.4\%, заработок таких обменных пунктов крайне велик. 

\item \emph{Какой может возникать риск?} У банка всегда «нулевая (или нейтральная) позиция», то есть, имея в начале дня определенные запасы валюты, к концу дня после процесса купли-продажи их количество почти не изменится (не займет позицию той или иной валюты). Если же банк продает слишком много одной валюты и покупает много другой, он рискует оказаться в провальной позиции, которая наступит при обесценивании закупленной валюты. Конкуренция же, обостряющаяся при завышении спреда, является, скорее, теоретической проблемой.

\item \emph{Как избавиться от этого риска?} Банк может продавать избыточные запасы валюты контрагенту, у которого более узкий спред (например, фьючерсами на срочных рынках, блокировкой гарантийного обеспечения). Тем самым он будет занимать противоположную (компенсирующую) позицию на другом рынке по такому же эксперименту.  
\end{itemize}

\section{Маркетмейкеры на финансовых рынках}

Маркетмейкинг (наравне с брокерским бизнесом) – единственный невероятностный, механический способ зарабатывать деньги на финансовых рынках. Доход в таком случае будет почти безрисковым. 

\subsection{Задача маркетмейкера}

Пусть биржа ММВБ (Московская Межбанковская Валютная Биржа) запускает новый контракт на срочном рынке (например, $ FSMICXZ9 $ – фьючерс на индекс ММВБ). Запуская торговый терминал, трейдер видит новый контракт и открывает котировку (купля-продажа). 

Будет ли кто-то там находиться в первый день? Если этот фьючерс на акцию есть на другой бирже, то достаточно ориентироваться на нее (потенциальный арбитраж для трейдера не создается). В общем же случае люди будут опасаться вставать в пустой стакан и выставлять запросы на покупку (неизвестность). Биржа же, очевидно, заинтересована в торговле контрактами. Поэтому она нанимает организации, которые выполняют функцию маркетмейкера. Они поддерживают двустороннюю пару котировок. 

Условно заключается договор с произвольным брокерским домом о том, что он будет держать 100 контрактов на покупку и 100 контрактов на продажу. Между ними устанавливается минимальный спред (регулируемый биржей) в 200 пунктов, и в течение дня брокерский дом должен поддерживать эту пару котировок (создавать ликвидность). За эту работу маркетмейкеру и платят некоторую фиксированную сумму (условно 150 тыс. руб./месяц за 1 фьючерс). В условиях договора прописано, что спред должен быть не меньше 200 пунктов, а объем в каждой котировке – не меньше 100 контрактов (в нашем примере). Теперь любой участник, открывая стакан, видит, что какие-то котировки есть и может воспринимать установленную маркетмейкером цену как адекватную (хотя в действительности она может быть неоправданной, то есть ошибочной). 
\subsection{Риски маркетмейкера}

Рассмотрим фьючерсы на акции. Например, $ FDGAZPZ9 $ – декабрьский фьючерс на «Газпром» (ММВБ), и соответствующий ему индекс РТС – $ GZZ9 $. Что делать маркетмейкеру, если ему ударили в оффер (после продажи открылась короткая позиция)? Он может начать поднимать пару котировок выше. Но ситуация будет повторяться, так как цена может продолжать расти (можем отслеживать рост по фьючерсам на других биржах), и с определенного момент начнутся убытки. 

Маркетмейкер вынужден перекрывать позицию в момент сделки. Это осуществляется с помощью робота на компьютере. Изначально он выставляет пару котировок на контракт (по формуле $ MidMarket \pm Spread $) и с помощью биржевого шлюза (без посредников) двигает ее, рассчитывая рыночную цену. Если же в пару ударяют, компьютер выполняет на РТС противоположное действие к тому, что произошло на ММВБ (на продажу отвечает покупкой, и наоборот). В итоге баланс сохраняется, так как контракты в обоих местах в целом идентичные (хотя их экспирация все-таки может отличаться). 

Касательно быстродействия торговых роботов. В связи с ростом их популярности многие отошли от кластеров (высокая стоимость) и перешли на видеокарты (многоядерные, заточены на большое количество операций линейной алгебры). В настоящее время акцент делается не столько на оптимизацию коммуникационных каналов, сколько на оптимизацию исполняющего софта. 

Риск заключается в том, что если, например, на РТС стоит всего 3 лота с узким спредом, то ничто не гарантирует его сохранение, если нам нужно 100 лотов (для нашего объема котировки). Но если спред на РТС будет больше, то мы всегда заведомо будем проигрывать. Поэтому важно осуществлять операцию перекрытия максимально быстро (до 500 мс), не допуская изменения цен. С другой стороны, если контракт, которым можно перекрыться с узким спредом, существует, возникает в чистом виде обменный пункт: всегда покупаем дешевле, а продаем дороже. Это можно назвать безрисковым заработком. Понятно, что в реальной жизни могут мешать сторонние факторы (к примеру, отключение света).

\subsection{Дополнительные проблемы}

Если говорить о мартовском фьючерсе ($ FDGAZPH0 $, $ GZH0 $), то на FORTS слишком большие ворота и маленькие объемы – им не получается перекрываться. Поэтому этот мартовский фьючерс на ММВБ приходится перекрывать декабрьским фьючерсом на РТС (работает в силу высокой корреляции между сериями фьючерсов). 

Биржа ужесточает требования для маркетмейкеров путем снижения спреда. С высоким значением спреда у маркетмейкеров без проблем получается убегать от сделок и двигаться в след за FORTS. Но поскольку физические лица стали жаловаться на это, биржа сократила спред в 2 раза. В связи с этим у маркетмейкров возрос риск не перекрыться. 

\section{Маркетмейкинг на фьючерсах на индекс}

Итак, с фьючерсами на акции все решается перекрытием эквивалентным контрактом. Но что делать с фьючерсами на индексы? Почему фьючерсы, например, на «Газпром» ($ GZZ9 $), на «Сбербанк» ($ SIZ9 $) торгуются в контанго (\textbf{Contango}), то есть цены на них выше акций – закладывается временная стоимость. В то же время фьючерс на индекс ($ RIZ9 $) будет торговаться в глубочайшем контанго (на 1 \% абсолютных). Это происходит от части потому, что индекс долларовый. Если бы фьючерсы на акции не торговались контанго, то возникал бы арбитраж (бесплатные деньги: если фьючерс торгуется дешевле, то можно продать акции в портфеле и купить фьючерс). Казалось бы, с индексом аналогично: с чего он должен торговаться бэквордацией? Однако индекс сложно восстановить. Фьючерсы на акции все поставочные, и их закрывают перед экспирацией. Фьючерс же на индекс является расчтеным – его нельзя поставить. Поэтому индекс допускает бэквордацию (\textbf{Backwardation}). 

Например, если купить 100 контрактов на РТС и продать контракты на «Газпром» и «Сбербанк» в соответствующих пропорциях (условно 30 и 25 соответственно) возникнет прямой арбитраж. На экспирации фьючерс на индекс сойдется к индексу, фьючерсы на акции сойдутся к акциям, а индекс равен сумме акций. Если расхождение сильное, можно заработать, заняв арбитражную позицию. 

\section{Как маркетмейкеры определяют цены фьючерса}

В реальности все зависит от того, как маркетмейкер будет хеджировать свои позиции. Например, если «Металлинвест» восстанавливает корзиной бумаг, он может держать фьючерс на уровне индекса (с нулевой ставкой). У него mid-market один. «АЛОР БРОКЕР» перекрывает часть на споте, часть на фьючерсах. У него другой расчет mid-market. Компания лектора хеджируется фьючерсом на индекс РТС. Третий принцип. И так далее. Второй и третий принципы смежны, и из-за этого пары лотов часто «бьются». Спред значительно снижается.  

При хеджировании фьючерсом индекса на ММВБ необходимо моделировать поведение фьючерса на РТС. Если же держать фьючерс на ММВБ в контанго, то может возникнуть следующая проблема. Пусть мы открыли длинную позицию (получили удар в бид (\textbf{Bid})), а на РТС - короткую. Если доллар ведет себя монотонно, ровно, то индексы сильно коррелированы. Однако фьючерс на РТС ведет себя хаотично и с течением времени уходит в контанго. И так как позиция короткая, базис увеличивается, возникает неотрицательная вариационная маржа, которая превышает относящуюся к фьючерсам на ММВБ. Хедж уже не является 100\%, в отличие от строящегося на акциях. 

Если хеджировать все сделки подряд долларом, то из-за больших ворот (за счет спреда) «сливаешь» еще больше. Поэтому долларовый риск обычно не хеджируется. Только если дельта изменения позиции в РТС составляет 40-50, это допустимо.

\emph{«Неужели ты не можешь заработать 1\% в день?»} – Если есть стопроцентная система прогнозирования, то к чему весь маркетмейкерский бизнес?

\section{Сколько и на чём зарабытывает маркетмейкер?}

В целом перекрываться на фьючерсах на акциях (держать спред шире) и зарабатывать на этом – реально. Однако с этого спреда никто не берет, так как физические лица не приходят. Поэтому в основном маркетмейкеры просто не создают арбитражных ситуаций: то есть не делают так, что их заявки можно купить дешевле, чем продать на FORTS. С индексом сложнее хеджироваться. Линейность не восстанавливается. Задача сводится к тому, чтобы отстоять свое время. Биржа вводит некоторую таксу (плату за время отстаивания). Маркетинг и формирование у клиентов привычки всегда видеть непустой стакан – ключевые фактора привлечения физических лиц.

С опционами работать гораздо тяжелее: сложнее, отсутствует линейность, больше степеней свободы (дополнительно необходимо считать кривую волатильности). Однако для опционов наблюдается меньшая динамика. 

\section{Как выглядели срочные рынки ММВБ и FORTS: межбиржевой арбитраж}

Раньше при открытии стаканов на ММВБ и FORTS можно было наблюдать сохранение одной и той же дельты между парами котировок в течение крайне продолжительного времени (до $ 10-15 $ секунд). На возникающем межбиржевом арбитраже можно было манипулировать контрактами с помощью написанного робота. Единственное, что требовалось контролировать, - это следить за тем, чтобы эта дельта не превышала выплачиваемую комиссию. Не требовались даже шлюзы - хватало торгового терминала. 

\section{Как выставляется заявка маркетмейкера с учётом спреда}

Пусть имеется $ FDGAZPZ9 $. Изучается стакан $ GZZ9 $, выбирается лучший оффер, берется половина от него (\textbf{mid-market}). Роботу передается значение параметра «чувствительность», «спред» (в основном пользователь робота меняет только этот параметр), «количество слотов». Относительно получившейся цены выставляется пара котировок вычитанием и прибавление заданного спреда. Далее после каждого изменения стакана рынок пересчитывается, и если новое значение mid-market'а отличается от предыдущего на значение «чувствительности», то пара котировок выставляется относительно нового mid-market. Значение спреда в роботе меняют в двух случаях: а) если кто-то встал с более узким спредом и начинает отбирать деньги у маркетмейкера, б) биржа установила новое значение спреда. 

\section{Попытки что-то предсказать по стакану и коварство больших лотов}

Попытки что-то предсказывать по стакану вряд ли увенчаются успехом. Стакан на ММВБ - отдельная таблица. Причем его глубина: 10/20 (ниже не видим). На FORTS же лежит бесконечная таблица заявок «Order State» с полями «номер-статус-цена-объем». Проходясь по этой таблице и вычленяя активные заявки, пользователь сам формирует стакан с сортировкой по цене.

Типичный пример ситуации с большими лотами. Кто-то выставил большое количество лотов (пусть 10,000). Для чего он это сделал? Пока люди скупают эти лоты, манипулятор незаметно снижает цену. Он наберет эти 10,000 слотов в короткую, а затем уберет заявку и начнет оказывать давление вниз. Более того, у любой биржи есть лимит изменения цены, который на фьючерсах соотносится с гарантийным обеспечением. 

\section{Лимитные и рыночные заявки}

В принципе, так как контракты, которыми хеджируются, известны, то можно использовать рыночные заявки. Альтернативой может выступать лимитная заявка с проскальзыванием 0.4\%. Проскальзывание (на какой процент повышаем цену выбранного оффера) позволяют учитывать возможное возрастание цены, которое произошло за время обработки слепка (считанного состояния) стакана. Оно позволяет существенно экономить время, затрачиваемое на перезапуск программы. В такой реализации лимитная заявка становится в некотором роде рыночной. Что касается комиссии биржи, она сильно занижены. Вдобавок биржа не берет комиссию за транзакции, а количество сделок крайне мало.

\end{document}
