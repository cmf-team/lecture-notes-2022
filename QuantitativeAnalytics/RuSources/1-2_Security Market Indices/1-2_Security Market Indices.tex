\documentclass{article}

% Language setting
% Replace `english' with e.g. `spanish' to change the document language
\usepackage[english,russian]{babel}
\usepackage{amsmath}

%графика
\usepackage{wrapfig}
\usepackage{graphicx}
\usepackage{pgfplots}
\usepackage{tikz}


\usepackage{tcolorbox}

% Set page size and margins
% Replace `letterpaper' with `a4paper' for UK/EU standard size
\usepackage[letterpaper,top=2cm,bottom=2cm,left=3cm,right=3cm,marginparwidth=1.75cm]{geometry}

% Useful packages
\usepackage{amsmath}
\usepackage{amssymb}
\usepackage{graphicx}
\usepackage{fixltx2e}
\usepackage[colorlinks=true, allcolors=blue]{hyperref}

\usepackage{geometry}
\geometry{left=25mm,right=25mm,
 top=25mm,bottom=25mm}

\title{Quantitative Analytics, Web3.\\
Lectures. Week 1-2. \\
Security Market Indices. \\
Рыночный индекс.}
\author{Georgii Gromov}

% Колонтитулы
\usepackage{fancyhdr}
\pagestyle{fancy}
\renewcommand{\headrulewidth}{0.1mm}  
\renewcommand{\footrulewidth}{0.1mm}
\lfoot{}
\rfoot{\thepage}
\cfoot{}
\rhead{CMF-2022}
\chead{}

\begin{document}
\maketitle

% Оглавление
\setcounter{tocdepth}{1} % {2} - в оглавлении участвуют chapter, section и subsection. {1} - только chapter и section
\renewcommand\contentsname{Contents}
\tableofcontents
\newpage

% \section{Dictionary, Definitions, Abbreviations}

% \subsection{Dictionary}
% \begin{itemize}
%     \item IR - Interest rate - процентная ставка.
%     \item Compounding - платежи (idk)
% \end{itemize}

% \subsection{Definitions and Abbreviations}
% \begin{itemize}
%     \item SAR - Stated annual rate.
%     \item EAR - Effective annual rate.
%     \item FoC - Frequency of Compounding
%     \item PMT - Payment
%     \item r - Interest rate (at the moment). 
% \end{itemize}

\renewcommand{\labelitemi}{\tiny$\bullet$}
\renewcommand{\figurename}{Fig.}

 \section{Определение}
 \begin{itemize}
     \item \textbf{Рыночный индекс} - показатель который отражает доходность какого-либо класса активов определенного рынка или сегмента этого рынка, он рассчитывается на основании изменения цены какого-либо набора бумаг. \\\\
    Например: индекс Московской биржи или индекс нефтегазовой отрасли.

 \end{itemize}


\section{Возвратный индекс (Index Return)}

Есть два основных возвратных индекса: \\\\
\textbf{Price Return Index} - отображает доходность стоимости ценных бумаг. \\\\
\textbf{Total return index} - помимо стоимости ценных бумаг включает в себя ещё и сумму дивидендов, то есть высчитывается как Price Return + Dividends. \\\\
\textbf{Обычно представляются Price Return индексы} \\\\
Некоторые компании для манипуляций сравнивают свой Total Return Index с Price Return Index конкурентов.



\section{Что нужно для того чтобы собрать и поддерживать индекс?}
\emph{Индекс - интегральный показатель, который рассчитывается по ценам акций входящих в его состав, но при этом доходность каждой акции мы можем брать с разным весом.}
\begin{enumerate}
  \item Определить какой рынок мы хотим характеризовать.
  \item Какие ценные бумаги мы хотим туда включить?
  \item Какие должны быть веса отдельных ценных бумаг? (самый распространенный вариант - чем больше капитализация компании, тем больше её вес)
  \item Как часто нужно ребалансировать индекс? (пересчитывать вес)
  \item Как часто индекс должен пересматриваться?
\end{enumerate}
Иногда вводят ограничения по весу.

\newpage

\section{Способы взвешивания бумаг в индексе}
Есть несколько основных способов определять вес бумаг в индексе: \\\\
\textbf{Взвешивание по цене акции (Price-weighted)}, в этом случае индекс представляет просто арифметическое среднее всех акций. \\
Формула  вычисления веса данного индекса:
$$ V_{pr.w} = \frac{\sum_{ i=1}^{N} n_iP_i}{D}$$
$D$ - корректируется с учетом разделения акций и изменений в портфеле \\\\
\textbf{Равновесное взвешивание (Equal-weighted)}.\\
У всех компаний в индексе будет иметься одинаковый вес.\\\\
\textbf{Взвешивание по размеру капитализации (Value-weighted or Market-cap-weighted)}\\\\
\textbf{Взвешивание по количеству акций в свободном обращении (Float-adjusted market-cap weighted)}. Один из примеров использования - индексы бумаг, обращающихся на определенных биржах, некоторые компании во время IPO размещают только часть акций, такой индекс как раз сделает поправку на такие ситуации.\\\\
\textbf{Взвешивание по фундаментальным показателям (Fundamental weighted)}



\section{Применения рыночного индекса}
\begin{itemize}
  \item Отражение настроения инвесторов
  \item Измерение качества работы управляющего активами
  \item Измерение средней рыночной доходности и риска
  \item Расчёт \href{https://en.wikipedia.org/wiki/Beta_(finance)}{показателя Бета}   
  \item Модельный портфель для индексных фондов.
\end{itemize}



\section{Какие классы индексов акций существуют}

\begin{itemize}
  \item \textbf{Широкие индексы рынка (Broad market index)}, например Willshire 5000, который включает топ 5000 американских компаний по капитализации и покрывает 99.5\% капитализации всех американских компаний торгующихся на бирже.
  \item \textbf{Межрыночные индексы (Multi-market index)}, например MSCI World, который включает акции компаний с рынков разных стран.
  \item \textbf{Межстрановые индексы с фундаментальными весами (Multi-market with fundamental weighting)}
  \item \textbf{Секторальные индексы (Sector index)}, например индекс электроэнергетики.
  \item  \textbf{Индексы по стилю (Style index)}, например индекс быстрорастущих компаний с небольшой прибылью \textbf{(Growth index)}. В противовес предыдущему классу акций - стоимостные акции, более стабильные, консервативные компании и менее быстрорастущие, но при этом платящие дивиденды и наращивающие прибыли, индекс таких акций будет называться - \textbf{(Value index)}. 
\end{itemize}
\newpage
\section{Параметры индексов облигаций}
\begin{itemize}
  \item \textbf{Страна} (Bloomberg Barclays US Aggregate Index, Euro C-Bonds Turkey)
  \item \textbf{Инвестиционный рейтинг} (Bloomberg Barclays US Corporate High-Yield, C-Bonds-CBI RU BB)
  \item \textbf{Длительность} (C-bonds-CBI RU 3-5 years)
  \item \textbf{Сектор} (S&P 500 Health Care Corporate Bond Index)
\end{itemize}
\section{Индексы альтернативных инвестиций}
\begin{itemize}
  \item \textbf{Индексы сырья}
  \item \textbf{Индексы рынка недвижимости}
  \item \textbf{Индексы хедж-фондов}
\end{itemize}

\end{document}
