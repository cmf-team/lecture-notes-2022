\documentclass{article}

% Language setting
% Replace `english' with e.g. `spanish' to change the document language
\usepackage[russian]{babel}
\usepackage{tcolorbox}

% Set page size and margins
% Replace `letterpaper' with `a4paper' for UK/EU standard size
\usepackage[letterpaper,top=2cm,bottom=2cm,left=3cm,right=3cm,marginparwidth=1.75cm]{geometry}

% Useful packages
\usepackage{amsmath}
\usepackage{amssymb}
\usepackage{graphicx}
\usepackage{fixltx2e}
\usepackage[colorlinks=true, allcolors=blue]{hyperref}

\title{Quantitative Analytics.\\
Lectures. Week 1. \\
The time value of money. Временная стоимость денег.}
\author{Chistyakov Artem}

\begin{document}
\maketitle

\textbf{
Lecture Topics
}

\begin{enumerate}
\item Interest rate and how it's formed.
\item SAR \& EAR.
\item FV, PV and FoC.
\item Annuities and perpetuity.
\item TVM problems.
\end{enumerate}



\section{Dictionary, Definitions, Abbreviations}

\subsection{Dictionary}
\begin{itemize}
    \item IR - Interest rate - процентная ставка.
    \item Compounding - платежи (idk)
\end{itemize}

\subsection{Definitions and Abbreviations}
\begin{itemize}
    \item SAR - Stated annual rate.
    \item EAR - Effective annual rate.
    \item FoC - Frequency of Compounding
    \item PMT - Payment
    \item r - Interest rate (at the moment). 
\end{itemize}


\section{Interpretation of the Interest rate}

\begin{itemize}
\item \textbf{Equilibrium rate of return}
\\
Minimum rate of return an investor must receive in order to accept the investment.

\item \textbf{The discount rate}
\\
Rate that must be applied to a cash flow to determine it's present value.

\item \textbf{The opportunity cost}
\\
Value that investor forgo (loses) by investing.

\end{itemize}



\section{How the Interest rate is formed}
\textbf{Interest rate =} \\
\textbf{Real risk-free rate} (Reflects the current vs future consumption)\\
+ \\
\textbf{Expected inflation premium} (Money costs less over time in a real terms)\\
+ \\
\textbf{Default risk premium} (Compensation for possibility and probability that the borrower will default) \\
+ \\
\textbf{Liquidity premium} (If financial product has a low liquidity - In case if you want to sell it quickly - you will be forced to take losses by selling it under the market price)\\
+ \\
\textbf{Maturity premium} (Long term debts are more sensitive to the future IR falls, if you expect such).

\section{SAR and EAR}
\textbf{SAR - Stated annual rate.} Rate which is stated in the documents.\\
\textbf{EAR - Effective annual rate.} Rate which an investor really gets.\\

\textit{SAR does not account for infra-year compounding while EAR does.}

\\

\textit{EAR \geq SAR}

\section{FV and PV}
\textbf{PV - Present Value.}\\
\textbf{FV - Future Value.}\\
\textbf{FV_x} \textbf{ - Future Value after x compounding periods (years by default)}\\

\[FV_x = PV * (1 + SAR)^x\]

\section{FV depends on FoC}

\textbf{Frequency of Compounding (FoC)} - how many times during the year the transaction happens.

\[FV_x = PV * (1 + \frac{SAR}{FoC})^{x*FoC} \]


\textnormal{and reversed}

\[PV = \frac{FV_x}{(1 + \frac{SAR}{FoQ})^{x*FoQ} } \]


\section{Annuities}
\textnormal{\textbf{Annuity is a list of \textit{identical} cash flows}}\\

\textnormal{Annuities can be}
\begin{enumerate}
    \item \textnormal{Finite}
    \begin{enumerate}
        \item \textnormal{\textbf{Ordinary annuity} - Cash flows occur \textbf{\textit{at the end}} of each \textbf{\textit{compounding period}}.}
        \item \textnormal{\textbf{Annuity due} - Cash flows occur \textbf{\textit{at the beginning}} of each \textbf{\textit{compounding period}}.}
    \end{enumerate}
    \item \textnormal{Infinite}
    \begin{enumerate}
    \item \textnormal{\textbf{Perpetuity.}}
    \end{enumerate}
    
\end{enumerate}


\section{PV of an annuity}


\[PV = PVofPMT_1 + PVofPMT_2 + \ldots + PVofPMT_n = \\
\frac{PMT}{1 + r} + \frac{PMT}{(1 + r)^2} + \ldots + \frac{PMT}{(1 + r)^n}\]
\\
\textnormal{We have a geometric sequence with}

\[b = \frac{PMT}{1 + r}\]
\\
\textnormal{and}
\\
\[a = \frac{1}{1 + r}\]
\\
\textnormal{So the sum equals to}
\\
\[S = b * \frac{q^n - 1}{q - 1}\]
\\
\textnormal{substitute b and q}
\\
\[PV\textsubscript{\textnormal{annuity}} = PMT * \frac{(1 + r)^n - 1}{r * (1 + r)^n}\]


\section{PV of a perpetuity}
\[PV\textsubscript{\textnormal{perpetuiry}} = \lim_{n \to +\infty} PV\textsubscript{\textnormal{annuity}} =  \lim_{n \to +\infty} PMT * \frac{(1 + r)^n - 1}{r * (1 + r)^n} = \frac{PMT}{r}\]

\end{document}
