\documentclass{article}

% Language setting
% Replace `english' with e.g. `spanish' to change the document language
\usepackage[english,russian]{babel}
\usepackage{amsmath}
\usepackage{pgfplots}

%графика
\usepackage{wrapfig}
\usepackage{graphicx}
\usepackage{pgfplots}
\usepackage{tikz}
\usepackage{stackengine}
\usepackage{eurosym}

\usepackage{tcolorbox}

% Set page size and margins
% Replace `letterpaper' with `a4paper' for UK/EU standard size
\usepackage[letterpaper,top=2cm,bottom=2cm,left=3cm,right=3cm,marginparwidth=1.75cm]{geometry}

% Useful packages
\usepackage{amsmath}
\usepackage{amssymb}
\usepackage{graphicx}
\usepackage{fixltx2e}
\usepackage[colorlinks=true, allcolors=blue]{hyperref}
\usepackage{caption} %заголовки плавающих объектов
\captionsetup[figure]{name=Рисунок}

\usepackage{geometry}
\geometry{left=25mm,right=25mm,
 top=25mm,bottom=25mm}

\title{Quantitative Analytics.\\
Lectures. Weeks 3-4. \\
Bonds trading.\\
Торговля облигациями.}
\author{Шарифуллин Александр}

% Колонтитулы
\usepackage{fancyhdr}
\pagestyle{fancy}
\renewcommand{\headrulewidth}{0.1mm}  
\renewcommand{\footrulewidth}{0.1mm}
\lfoot{}
\rfoot{\thepage}
\cfoot{}
\rhead{CMF-2022}
\chead{}

\begin{document}
\maketitle

% Оглавление
\setcounter{tocdepth}{1} % {2} - в оглавлении участвуют chapter, section и subsection. {1} - только chapter и section
\renewcommand\contentsname{Содержание}
\tableofcontents
\newpage

% \section{Dictionary, Definitions, Abbreviations}

% \subsection{Dictionary}
% \begin{itemize}
%     \item IR - Interest rate - процентная ставка.
%     \item Compounding - платежи (idk)
% \end{itemize}

% \subsection{Definitions and Abbreviations}
% \begin{itemize}
%     \item SAR - Stated annual rate.
%     \item EAR - Effective annual rate.
%     \item FoC - Frequency of Compounding
%     \item PMT - Payment
%     \item r - Interest rate (at the moment). 
% \end{itemize}

\renewcommand{\labelitemi}{\tiny$\bullet$}
\renewcommand{\figurename}{Fig.}
\def\Ruble{\stackengine{.64ex}{%
  \stackengine{.4ex}{\textbf{\textsf{P}}}{\rule{1ex}{.16ex}\kern.55ex}{O}{r}{F}{F}{L}%
  }{\rule{1ex}{.16ex}\kern.55ex}{O}{r}{F}{F}{L}\kern-.1ex}
\def\yenrule{\rule{1.3ex}{.1ex}}
\def\textyen{\renewcommand\stacktype{L}\stackon[.4ex]{\stackon[.65ex]{Y}{\yenrule}}{\yenrule}}


\section{Разница инвестиционных и коммерческих банков}
\textbf{Инвестиционный банк} -- финансовый институт, который организует для крупных компаний и правительств привлечение капитала на мировых финансовых рынках, а также оказывает консультационные услуги при покупке и продаже бизнеса, брокерские услуги, являясь посредником при торговле акциями и облигациями, производными финансовыми инструментами, валютами и сырьевыми товарами, а также выпускает аналитические отчеты по всем рынкам, на которых он оперирует. Основной инструмент получения прибыли -- размещение финансовых активов таким образом, чтобы осуществлять дешёвую закупку активов и их дорогую продажу.\\
\textbf{Коммерческий банк} -- кредитная организация, осуществляющая банковские операции для юридических и физических лиц, основным источником дохода которой служат разница в процентных ставках между выдаваемыми кредитами и размещаемыми депозитами (вкладами) и комиссионные сборы.\\
\section{Основные направления деятельности инвестиционных банков}
Выделяют следующие основные направления деятельности инвестиционных банков:
\begin{itemize}
    \item \textbf{transaction banking} -- проведение сделок по выпуску новых инструментов на рынки (первичное размещение, к которому относятся: слияние/поглощение, IPO, выпуск облигаций) ECM (equity capital markets) и DCM (debt capital markets), доходность за счёт коммиссий;
    \item \textbf{sales and trading} -- вторичные операции на ECM и DCM по торговле финансовыми инструментами;
    \item \textbf{research} -- направление по формированию аналитики и финансовой отчётности для компаний с целью привличения новых клиентов на первые два направления.
\end{itemize}
\section{Доходности финансовых инструментов. Доходности долговых инструментов}
\begin{figure}[h]
\begin{center}
    \begin{tikzpicture}
        \begin{axis}[
        	title = График доходность-риск,
        	xlabel = {$r$},
        	ylabel = {$\sigma$},
        	xticklabels={,,},
        	yticklabels={,,}
            ]
            \node[] at (axis cs: .9,.9) {A};
            \node[] at (axis cs: 4.1,4.25) {B};
            \addplot[blue] coordinates {(1,1) (2,2.85) (3,3.75) (4,4.15)};
        \end{axis}
    \end{tikzpicture}
\end{center}
\caption{График доходность-риск для различных финансовых инструментов; по осям r - доходность, $\sigma$ - риск: A - долговой актив (облигации), B - долевой актив (акции)}
\label{pic1_r_sigma_plot}
\end{figure}
Доходности долговых финансовых инструментов (актуально на момент выхода лекции):\\
\begin{table}[]
    \centering
    \begin{tabular}{c|c|c|c|c}
         & \$ & \euro & \Ruble & \textyen \\
        3 года & 5\% & 4\% & >8\% & <2\%
    \end{tabular}
    \caption{Доходности по долговым финансовым инструментам в различных валютах}
    \label{tab1_bonds_returns}
\end{table}\\

\textit{Пояснение про йену}: в Японии политика "дешёвых" денег, ставки занижают, чтобы стимулировать кредитование, но всё равно сохраняется большая дефляция.\\
\section{Различие кредитов и облигаций}
\textbf{Облигации} обращаются на финансовых рынках. \textbf{Кредиты} не обращаются, выдаются лицам напрямую.\\
Коммерческий банк -- 80-95\% кредитов, 5-20\% облигаций (в допущении оптимума по Парето и отсутствия акций). На самом деле, 5\% -- наиболее оптимальное значение, ниже пример, почему большие значения являются рисковыми для банка.\\
\textit{Пояснение про закрытие коммерческих банков в РФ}: в РФ в середине 10-х происходило закрытие банков и отзыв лицензий. В среднем, отношение собственных средств к активам в банке составляет ~10\%. Когда произошёл обвал котируемых цен облигаций до 30-60\% от номинала, произошло падение их капитализации в 2 раза. В момент погашения банк обязуется выплатить полную (номинальную) стоимость облигаций (т. е. 100\%), поэтому потеря половины стоимости облигаций привела к необходимости выплаты всех имевшихся у закрывшихся банков 10\% собственных средств, то есть капиталы банков обнулились.\\
На что влияет часть средств банка в активах:
\begin{itemize}
    \item \textit{PnL} -- прибыль банка;
    \item \textit{CAR} -- капитал банка.
\end{itemize}
\section{Типы эмитентов облигаций}
Облигации делятся на:
\begin{itemize}
    \item суверенные (государственные);
    \item корпоративные;
    \item муниципальные;
    \item региональные;
    \item supernational или межнациональные.
\end{itemize}
\section{Рейтинги облигаций}
Облигации можно рейтинговать по надёжности эмитента. Чем выше надёжность, тем стабильнее облигация, меньше шансов банкротства эмитента.
Рейтинги (на момент выхода лекции):
\begin{itemize}
    \item AAA -- например, облигации США, Великобритании, Германии, среди корпоративных - MCD (Макдональдс);
    \item AA
    \item A
    \item BBB -- например, РФ;
    \item BB
    \item B
\end{itemize}
На сроке в 5 лет разница между AAA и BBB разница составляет 2-5\% в стоимости.
\section{Основные участники рынка облигаций}
\begin{itemize}
    \item ФНБ(фонд национального благосостояния)/SNF/ЦБ -- самые крупные участники рынка облигаций, основная цель управления -- максимальное уменьшение риска и сохранение денег;
    \item банки, основная цель управления -- сохранение и доходность (return equity) > 30\%;
    \item управляющие компании (деньги пенсионные, страховые и т.д.), основная цель -- сохранение и доходность; повышение индекса EMBI;
    \item хедж-фонды -- достижение и преодоление перформанса рынка акций.
\end{itemize}
\section{Специфика рынка облигаций развивающихся стран (emerging markets or EM)}
Повышены следующие риски:
\begin{itemize}
    \item кредитный;
    \item ценовой;
    \item ликвидности.
\end{itemize}
\section{Производные финансовые инструменты}
Производные: фьючерсы, опционы, cds (certificates of deposites), свопы.
ISDA -- international swaps and derivatives association. Разработала документацию для разных финансовых продуктов, принятую во всём мире (кроме РФ и пр.). Основывается на концепции netting (взаимозачёта встречных требований).\\
В силу того, что в РФ это не работает, то рынок производных финансовых инструментов по облигациям в РФ не развит.\\
Фьючерсы -- цены наблюдаемы, ценообразование понятно.
Опционы -- большая часть торгуется как внебиржевые решения, формирование цен описывается затруднительно.\\
Свопы -- хотя и внебиржевые, их цены публикуются, что позволяет понять процесс их ценообразования.\\
Концепция работы со сложными деривативами -- разбиение на более простые и работа с ними таким образом, чтобы PnL был > 0 (концепция дельта-хеджирования).\\
\section{О кредитном качестве, дефолтах и реструктуризации}
При падении кредитного качества цены облигации не падают сразу в ноль. Ноль -- совсем полное банкротство.\\
Невыплата купона может быть по причинам:
\begin{itemize}
    \item падения ликвидности;
    \item падения доходности;
    \item реструктуризации долга.
\end{itemize}
Поэтому скорее при сильном падении стоимости облигаций происходит реструктуризация, банкротство -- более редкий сценарий.
\section{Работа с облигациями в инвестиционном банке}
В описанном в лекции отделе 5 человек. Два человека предоставляют ликвидность на рынке клиентам банка. Третий занимается собственной позицией банка на рынке. Четвёртый занимается поиском новых продуктов. Пятый занимается функцией контроля качества работы участников отдела.
\section{Источники аналитики по облигациям}
Принцип "Китайской стены": есть public и private side. Люди на public side используют только публичную информацию, люди на private side используют информацию внутри компаний. При этом обмен информации не допустим (для недопущения инсайдерской торговли). Опираются при этом риск-менеджеры на публичную информацию от аналитиков (то есть на public side), а также на алгоритмические стратегии.
\section{Сейлзы ("отдел продаж") по облигациям}
Клиент звонит в сейлз, сейлз обращается в отдел по торговле облигациями с запросом котировки и размещения заявки клиента.\\
Обратное работает также -- трейдеры со стороны банка определяют, какие облигации нужно продать и размещают заявку для сейлза по поиску клиентов для продажи.
\section{Количественная аналитика по облигациям}
\begin{itemize}
    \item построение кривой discount factor r(t);
    \item оценка облигации по формуле:
    \[PV = \frac{cf_T}{(1+r_t)^T}\]
\end{itemize}
Ноу-хау в оценке облигации -- умение оценить переменные, которых нет, по тем переменным, которые есть.
\end{document}
