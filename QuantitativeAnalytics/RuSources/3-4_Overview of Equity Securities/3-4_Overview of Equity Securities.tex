\documentclass{article}

% Language setting
% Replace `english' with e.g. `spanish' to change the document language
\usepackage[english,russian]{babel}
\usepackage{amsmath}

%графика
\usepackage{wrapfig}
\usepackage{graphicx}
\usepackage{pgfplots}
\usepackage{tikz}


\usepackage{tcolorbox}

% Set page size and margins
% Replace `letterpaper' with `a4paper' for UK/EU standard size
\usepackage[letterpaper,top=2cm,bottom=2cm,left=3cm,right=3cm,marginparwidth=1.75cm]{geometry}

% Useful packages
\usepackage{amsmath}
\usepackage{amssymb}
\usepackage{graphicx}
\usepackage{fixltx2e}
\usepackage[colorlinks=true, allcolors=blue]{hyperref}

\usepackage{geometry}
\geometry{left=25mm,right=25mm,
 top=25mm,bottom=25mm}

\title{Quantitative Analytics.\\
Lectures. Week 4. \\
OVERVIEW OF EQUITY SECURITIES. Обзор ценных бумаг в виде акций.}
\author{Basaeva Nayana}

% Колонтитулы
\usepackage{fancyhdr}
\pagestyle{fancy}
\renewcommand{\headrulewidth}{0.1mm}  
\renewcommand{\footrulewidth}{0.1mm}
\lfoot{}
\rfoot{\thepage}
\cfoot{}
\rhead{CMF-2022}
\chead{}

\begin{document}
\maketitle

% Оглавление
\setcounter{tocdepth}{1} % {2} - в оглавлении участвуют chapter, section и subsection. {1} - только chapter и section
\renewcommand\contentsname{Оглавление}
\tableofcontents
\newpage

% \section{Dictionary, Definitions, Abbreviations}

% \subsection{Dictionary}
% \begin{itemize}
%     \item IR - Interest rate - процентная ставка.
%     \item Compounding - платежи (idk)
% \end{itemize}

% \subsection{Definitions and Abbreviations}
% \begin{itemize}
%     \item SAR - Stated annual rate.
%     \item EAR - Effective annual rate.
%     \item FoC - Frequency of Compounding
%     \item PMT - Payment
%     \item r - Interest rate (at the moment). 
% \end{itemize}

\renewcommand{\labelitemi}{\tiny$\bullet$}
\renewcommand{\figurename}{Fig.}

 \section{Основные свойства акций. Система голосования.}
     \textbf{Акция} - вид долевой ценной бумаги, обладающей следующими свойствами:
     
\begin{itemize}
     \item дает владельцу право на получение части чистого дохода от деятельности акционерного общества в виде дивидендов

     \item дает владельцу право на участие в собрании акционеров и голосование по основным вопросам деятельности компании

     \item дает владульцу право на часть имущества компании в случае ее ликвидации
     \end{itemize}
     
     На практике не всегда все свойства выполняются. Например, компания Yandex не выплачивает дивиденды, а предпочитает их инвестировать в новые проекты. Право на часть остаточной стоимости подразумевается в случае банкротства компании, т. е. нельзя просто прийти и обменять акцию на часть активов. 
     
     При обладании недостаточно большим пакетом акций владелец не сможет оказывать влияние на принятие решений. Чтобы решить проблему голосования, была изменениа система корпоративного управления. В следсвтие этого различают следующие типы систем голосования: 
     \begin{itemize}
     \item Statutory voting - раздельное (прямое) голосование. 1 акция = 1 голос. Кождый голос используется при голосовании по каждому вопросу. Данная система позволяет мажоритарному акционеру принимать единолично любые решения, так как по кажому вопросу у него всегда будет большинтсвот голосов. (Пример: имея 60 процентов голосов, можно на каждое голосование отдавать по 60 процентов и выигрывать их). 

     \item Cumulative voting - акционер использует общее количество своих голосов по всем вопросам сразу, то есть распределяет их между вопросами. (Пример: имея 60 голосов, можно 20 отдать на первое голосование, тогда на последующие уже останется 40). Данная система позволяет миноритариям объединяться и проводить своего независимого кандидата в совет директоров и влиять на решения.
\end{itemize}

Помимо обыкновенных акций (common shares) также существуют \textbf{акции со встроенными опционами}.

\textbf{Callable common shares} - акции, которые могут быть выкуплены (отозваны) компанией по определенной цене.

\textbf{Putable common shares} - акции, которые дают  инвестору возможность продать акции компании обратно по опредленной цене.

\section{Привилегированные акции}

    \textbf{Привилегированные акции (preferred stock)} - акции, гарантирующие инвестору фиксированные дивиденды, но лишающие инвестора права голоса.

\textbf{Кумулятивные привилегированные акции (cumulative preferred stock)} - привилегированные акции, которые дают владельцу право получить дивиденды за прошлые годы, не выплаченные по каким-либо причинам. Компании не обязаны платить дивиденды по привилегированным аациям, если доходы за данный период оказались недостаточными. Кумулятивные привилегированные акции гарантируют, что если компания в последующие годы начнет снова приносить прибыль, эти неуплаченные дивиденды будут выплачены прежде, чем начнется выплата дивидендов по обыкновенным акциям.

\textbf{Конвертируемые привилегированные акции (convertible preferred stock)} - привилегирогванные акции, которые в определенных ситуациях могут быть сконвертированы в обыкновенные акции с установленным коэффициентом. 


\section{Причины оставаться частной компанией}

\textbf{Публичные компании } - компании, акции которых обращаются на бирже.

\textbf{Частные компании } - компании, акции которых можно купить только у текущих акционеров.

Причины, по которым компании остаются частными:

\begin{itemize}
     \item Менее ликвидные
     \item Сделки на частном рынке происходят путем переговоров (каждая из сторон проводит собственный анализ)
     \item Отсутствие требования публикации финансовой отчетности в открытом доступе (что дает возможность не раскрывать особенности бизнеса конкурентам)
     \item Более низкие затраты на отчетность, то есть отсутствие необходимости содержать большой штат бухгалтерии, финансовых аналитиков
     \item Более слабые стандарты корпоративного управления (следует из предыдущего пункта: нет надзора, нет проверки со стороны независимых инвесторов)
     \item Возможность концентрироваться на долгосрочных результатах. Отсутсвие обязательства публиковать отчестности позволяет принимать не самые выгодные в кроткосрочном периоде решения, но окупающиеся долгосрочно.
\end{itemize}

В ранках частных инвестиций существуют 3 основные категории инвестиционных фондов, которые занимаются соотвествченными категориями сделок: 
\begin{itemize}
     \item Венчурный фонд (VC - Venture Capital) - самые рискованные инвестиции, стартапы 
     \item "Финансируемый выкуп", "Выкуп засчет кредита" (LBO - Leveraged Buy-Out) -  выкуп уже состоявшейся компании
     \item Фонд прямых частных инвестиций, частный капитал (PE - Private Equity) - ищут на частном рынке подходящие для них компании и выкупают целиком
\end{itemize}


\section{Зарубежный капитал}

Существует проблема инвестиций в зарубежные рынки. Одним из решений этой проблемы является \textbf{депозитарная расписка (depository receipt)} 

\textbf{Депозитарная расписка (Depository receipt)} - ценная бумага, которая позволяет вкладывать деньги в иностранные ценные бумаги, оставаясь на привычной бирже, через своего брокера.

Основные типы расписок:
\begin{itemize}
     \item GDR (Global depository receipt) - расписка, которая обращается не в США и не в домашней стране. 
     \item ADR (American depository receipt) - расписка, которая обращается в США. 
\end{itemize}



\section{Уровень риска инвестиций}
От наиболее к наименее рисковым:
\begin{itemize}
     \item Callable shares. Падение акций неограничено (риск), а рост ограничен (прибыль)
     \item Common stock 
     \item Preferred stock. Меньше рисков засчет фиксированных дивидендов
     \item Cumulative preferred shares. Даже если какие-то дивиденды будут пропущены, компания должна будет их доплатить в будущем 
\end{itemize}


\section{Балансовая и рыночная стоимости акций}

\textbf{Балансовая стоимость акции (Book value)} - капитал компании, разница между активами и пассивами.

\textbf{Рыночная стоимость акции (Market value)} - капитализация компании: число акций, умноженное на цену одной акции

Таким образом балансовая стоимость - фактическая стоимость по финансовой отчетности, отражает результаты деятельности компании в прошлом. Рыночная стоимость зависит от большого числа ожиданий инвесторов, отражает ожидание о будущих денежных потоках.

\section{Return on equity and cost of equity}
$$ROE_{t} =\frac{NI_{t}}{BV_{t-1}}$$

$ROE_{t}$ - return on equity - доходность на вложенный капитал

$NI_{t}$ - net income - чистая прибыль

$BV_{t-1}$ - book value - балансовая стоимость акции

Return on equity - фактическая стоимость доходности, которую компания заработала за определенный период. 

Cost of equity - ожидаемая величина, котррая показывает равновесную жоходность на акции этой компании. Зависит от риска и восприятия компании инвестором. Чем выше риск, тем больше дохода ожидает получить инвестор, тем выше показатель.

\end{document}
