\documentclass{article}
% Language setting
% Replace `english' with e.g. `spanish' to change the document language
\usepackage[english,russian]{babel}
\usepackage{amsmath}

%графика
\usepackage{wrapfig}
\usepackage{graphicx}
\usepackage{pgfplots}
\usepackage{tikz}

\usepackage{tcolorbox}

\usepackage[letterpaper,top=2cm,bottom=2cm,left=3cm,right=3cm,marginparwidth=1.75cm]{geometry}

\usepackage{amsmath}
\usepackage{amssymb}
\usepackage{graphicx}
\usepackage{fixltx2e}
\usepackage[colorlinks=true]{hyperref}

\usepackage{geometry}
\geometry{left=25mm,right=25mm,
 top=25mm,bottom=25mm}

\title{Quantitative Analytics.\\
Lectures. Weeks 5 - 8. \\}
\author{Tolmachev Daniil}

\usepackage{fancyhdr}
\pagestyle{fancy}
\renewcommand{\headrulewidth}{0.1mm}  
\renewcommand{\footrulewidth}{0.1mm}
\lfoot{}
\rfoot{\thepage}
\cfoot{}
\rhead{CMF-2022}
\chead{}

\begin{document}
\maketitle

\setcounter{tocdepth}{1} 
\renewcommand\contentsname{Оглавление}
{
  \hypersetup{linkcolor=blue}
  \tableofcontents
}
\newpage

\section{FX forwards}
Два вида форвардов:
\begin{itemize}
\item
Поставочный (deliverable) форвард
\item
Беспоставочный (non-deliverable) форвард, NDF
\end{itemize}
Форвардный курс на каждую конкретную дату ($F$) - это расчетная величина, которая зависит от:
\begin{itemize}
\item
Курса спот\qquad\qquad\qquad\qquad\quad\; : S
\item
Процентной ставке по валюте 1 : $I_{dc}$
\item
Процентной ставке по валюте 2 : $I_{fc}$
\item
срока сделки\qquad\qquad\qquad\qquad\quad : T
\end{itemize}
Формула расчета в упрощенном виде : 
$$ F = \frac{(1+I_{dc})^T}{(1+I_{fc})^T}S $$
\subsection*{Кредитный риск (Credit Exposure)}
\textbf{Пример.}

Мы покупаем форвард у контрагента, цена растет (для нас/контрагента переоценка положительна/отрицательна). Цена может вырасти до такого значения,  что контрагент не сможет (или откажется) выполнять свои обязательства по контракту, и мы можем понести убытки. 

\href{https://www.investopedia.com/terms/c/credit-exposure.asp}{(Не из лекции)} \textbf{Кредитный риск} — это величина максимальных потенциальных убытков для кредитора в случае невыполнения заемщиком своих обязательств по платежу.

Кредитный риск по деривативам рассчитывается как сумма текущей переоценки и будущего риска. \href{https://youtu.be/ev2jSbH9G9U?t=952}{Пример расчет кредитного риска на годовом форварде.}
\section{Валютный опцион}
В случае продажи опциона клиенту, бане не несет риск на контрагента. В случае покупки опциона у клиента риск  аналогичен риску по форварду и лимит должен рассчитываться также.
\section{Процентный своп}
\textbf{Процентный своп} - это сделка, в которой стороны обмениваются потоками процентных платежей. Обычно обмениваются платежи по фиксированной ставке на платежи по переменной ставке.
Для оценки свопа используют оценки бондов с фиксированной и с плавающей ставками.
\href{https://youtu.be/ev2jSbH9G9U?t=1789}{Пример расчет кредитного риска по свопу.}\\
Типы управления кредитным риском:
\begin{itemize}
\item 
Лимитирование - для контрагента устанавливается лимит по кредитному риску по деривативным операциям;
\item
Элиминирование - контрагент предоставляет обеспечение в размере кредитного риска.
\end{itemize}
\section{Cross-currency Swap}
\textbf{Cross-currency swap} - сделка, в которой стороны обмениваются потоками процентных платежей в разных валютах. Обычно обмениваются платежи по фиксированной ставке на платежи по переменной ставке. При таких сделках появляются еще риски, связанные с курсами валют.
\section{Подходы к оценке рисков деривативов. Оценка VaR}
\begin{itemize}
\item
Delta-Normal
\item
Grid-search
\item
Monte-Carlo simulations
\end{itemize}
Рассмотрим на примере.  Предположим, что мы купили одномесячный 
Spot 68.4\$; Strike 68.4\$; Implied volatility 50\%; Notional 10'000 shares\\
Delta(shares) 5'397; Delta (\$) 370'000; Volatility of the spot price 90%.
\subsection{Delta-Normal}
$$\text{One-day } \text{VaR}_{1-\alpha} = q_{1-\alpha}\cdot Delta\cdot P\cdot\sigma$$
Для нашего случая:
$$\text{One-day } \text{VaR}_{99\%} = 3\cdot5397\cdot68.4\cdot\frac{90\%}{\sqrt{260}} = 61814\; \$ $$
Минусы такого подхода:
\begin{itemize}
\item[--]
Delta эквивалент работает только в первом приближение;
\item[--]
Эффекты волатильности и процентных ставок не учитываются.
\end{itemize}
\subsection{Grid-search}
Подход использует полную переоценку
позиции деривативов для различных
сценариев базовых переменных (underlying asset spot price, implied volatility, interest rate), где каждая переменная изменяется в
заранее определенном диапазоне (например, +/- 3 Std Dev). Недостаток метода заключается в том, что корреляции между переменными не учитываются.\\ \href{https://youtu.be/ev2jSbH9G9U?t=2789}{Расчеты для нашего примера.}
Результат: $\text{One-day } \text{VaR}_{99\%} = 35989 \;\$ $
\subsection{Monte-Carlo}
Метод Монте-Карло использует полную переоценку позиции деривативов для различных сценариев моделирования базовых переменных с учетом корреляций между ними.\\


\textbf{Методология:}
\begin{enumerate}
\item
Скорректировать доходность базовых активов с учетом тренда волатильности (GARCH-методы);
\item
Смоделировать доходность базовых
активов с поправкой на волатильность, случайным образом выбирая их из выборки;
\item
Смоделировать изменение безрисковой ставки;
\item 
Смоделировать implied волатильности (GARCH-методы);
\item
Переоценить позицию, используя смоделированные переменные, и
взять квантили результирующего ряда P\!\&\!L.
\end{enumerate}
\textbf{Корректировка трендов волатильности:}
\begin{enumerate}
\item
Оценить параметры GARCH-модели для доходности
базовых активов;
\item 
Поделить каждую доходность на соответствующее
стандартное отклонение GARCH и получить стандартизированный возврат;
\item
Умножить стандартизированную доходность на
прогнозируемую волатильность, получив доходность с поправкой на волатильность
\end{enumerate}
\textbf{Моделирование implied volatility:}
\begin{enumerate}
\item
Случайным образом выбрать доходность с поправкой на волатильность из выборки;
\item 
Рассчитать волатильность с помощью GARCH модели;
\item
Рассчитать среднюю волатильность за необходимый
моделируемый период.
\end{enumerate}

\href{https://youtu.be/ev2jSbH9G9U?t=3379}{Расчеты для нашего примера.} Результат: $\text{One-day } \text{VaR}_{99\%} = 32559 \;\$ $
\subsection{Вывод}
\begin{enumerate}
\item
Дельта-нормальный подход в общем дает
неточные оценки VaR (особенно для
опционов ATM);
\item
Подход с сеточным поиском прост в реализации,
однако он не учитывает корреляцию
между переменными и различными активами;
\item
Метод Монте-Карло дает более точные оценки
VaR. Однако сложен   в реализации.
\end{enumerate}
Подробнее об оценках VaR можно почитать в этой статье Pilar Abad, Sonia Benito, Carmen López \\"A comprehensive review of Value at Risk methodologies"  
\section{Стресс-тестирование}
Целью стресс-тестов является оценка влияния стрессовых событий, вероятность которых очень мала.

Типы стресс-тестирования:
\begin{enumerate}
\item
Анализ чувствительности (рассматривается влияние изменения одного из факторов риска) 
\item
Сценарный анализ (рассматривается изменение сразу нескольких факторов риска)
\end{enumerate}

%Типы стресс-тестирования:
%\begin{enumerate}
%\item
%Макро стресс-тестирование
%\item
%Микро стресс-тестирование
%\end{enumerate}

Исторические данные показывают, что во время падения рынков корреляции растут. Поэтому в простых стресс-тестах корреляции выбираются равными 1.

Grid-search метод применяется для стресс-тестирования. Диапазон переменных выбирается +/- 10 стандартных отклонений или с учетом исторических данных стрессовых событий. Опять же возникает проблема учета корреляций.
\end{document}